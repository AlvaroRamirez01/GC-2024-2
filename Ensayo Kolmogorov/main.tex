\documentclass[a4paper, 12pt]{article}

% Paquetes necesarios
\usepackage[utf8]{inputenc} % Codificación de caracteres
\usepackage[spanish]{babel} % Idioma del documento
\usepackage[left=2.5cm, right=2.5cm, top=2cm, bottom=2cm]{geometry} % Márgenes
\usepackage{lipsum} % Generador de texto de ejemplo, puedes eliminarlo en tu ensayo final

% Título
\title{El hombre que invento la probabilidad moderna}
\author{Alvaro Ramírez López}
\date{\today} % Puedes modificarlo si deseas una fecha específica

\begin{document}

\maketitle

% Introducción
\section{Introducción}
Aquí es donde presentas tu tema y propósito del ensayo. Puedes definir términos clave y explicar por qué es importante el tema que estás discutiendo.

% Cuerpo del ensayo
\section{Desarrollo}
Esta sección contiene tu argumento principal. Puedes dividirla en varias subsecciones si es necesario. Asegúrate de presentar evidencia y ejemplos para respaldar tus puntos.

\subsection{Subsección 1}
Aquí puedes desarrollar un aspecto específico de tu argumento.

\subsection{Subsección 2}
Otro aspecto que deseas abordar.

% Conclusión
\section{Conclusión}
Resumen de tus argumentos y presentación de tus conclusiones finales. Puedes destacar la importancia de tu investigación y ofrecer sugerencias para investigaciones futuras.

% Referencias
\begin{thebibliography}{}
\bibitem{ejemplo1} Autor1. Título del Libro. Editorial, Año.
\bibitem{ejemplo2} Autor2. Título del Artículo. Revista, Volumen (Número), Páginas, Año.
\end{thebibliography}

\end{document}
